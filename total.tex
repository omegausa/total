\documentclass[11pt]{scrartcl}
%% PREAMBLE (for distribution or copy-pasting)

%%% Load packages
\usepackage{amsmath,amssymb,amsthm}
\usepackage[usenames,dvipsnames,svgnames]{xcolor}
\usepackage{hyperref}
\usepackage{mathtools}
\usepackage{microtype}

%%% Colored sections
\renewcommand*{\sectionformat}%
  {\color{purple}\S\thesection\enskip}
\renewcommand*{\subsectionformat}%
  {\color{purple}\S\thesubsection\enskip}
\renewcommand*{\subsubsectionformat}%
  {\color{purple}\S\thesubsubsection\enskip}
\addtokomafont{paragraph}{\color{orange!35!black}\P\ }
\KOMAoptions{numbers=noenddot}

%%% Title
\addtokomafont{subtitle}{\Large}
\setkomafont{author}{\Large\scshape}
\setkomafont{date}{\Large\normalsize}

%%% Page setup
\usepackage[headsepline]{scrlayer-scrpage}
\renewcommand{\headfont}{}
\addtolength{\textheight}{3.14cm}
\setlength{\footskip}{0.5in}
\setlength{\headsep}{10pt}
\makeatletter
\ihead{\footnotesize\textbf{\@author} (\@date)}
\automark{section}
\chead{}
\ohead{\footnotesize\textbf{\@title}}
\cfoot{\pagemark}
\makeatother

%%% Macros
\providecommand{\ol}{\overline}
\providecommand{\ul}{\underline}
\providecommand{\wt}{\widetilde}
\providecommand{\wh}{\widehat}
\providecommand{\eps}{\varepsilon}
\providecommand{\half}{\frac{1}{2}}
\providecommand{\inv}{^{-1}}
\newcommand{\dang}{\measuredangle} %% Directed angle
\providecommand{\CC}{\mathbb C}
\providecommand{\FF}{\mathbb F}
\providecommand{\NN}{\mathbb N}
\providecommand{\QQ}{\mathbb Q}
\providecommand{\RR}{\mathbb R}
\providecommand{\ZZ}{\mathbb Z}
\providecommand{\ts}{\textsuperscript}
\providecommand{\dg}{^\circ}
\providecommand{\ii}{\item}
\providecommand{\defeq}{\coloneqq}
\DeclareMathOperator*{\lcm}{lcm}
\DeclareMathOperator*{\argmin}{arg min}
\DeclareMathOperator*{\argmax}{arg max}
\providecommand{\hrulebar}{\par
\hspace{\fill}\rule{0.95\linewidth}{.7pt}\hspace{\fill}
\par\nointerlineskip \vspace{\baselineskip}}

%%% Optional colored theorems
\usepackage{thmtools}
\declaretheorem[style=definition,name=Theorem,numberwithin=section]{theorem}

%% END PREAMBLE

\begin{document}
\title{Trivium of Theorems and Lemmas}
\subtitle{American Masters in Math}
\author{OMEGA}
\date{February 2026}
\maketitle

\tableofcontents

\section{Preface}
\subsection{Motivation}
In math competitions, there is often significant confusion or ``gray areas''
in what results are allowed to cite on competitions.
The aim of the TOTAL is to resolve that ambiguity, as follows:

\begin{itemize}
  \item Every theorem that appears in TOTAL may be quoted on our exams.
    We expect this to cover the extreme majority of use cases.
  \item On rare occasions, we may choose to disallow certain results
    (see \autoref{sec:forbidden}).
  \item In the case of results which are neither allowed nor forbidden,
    the discretion is up to the organizers.
    Our general guidance is to accept named theorems that do not trivialize the problem.
    However, we hope that this situation will be extremely rare,
    and to not occur in a typical year.
\end{itemize}

\subsection{What this TOTAL is not}
The TOTAL is \emph{not} the following:
\begin{itemize}
  \item This is not a list of theorems we expect everyone to know.

  In fact, if a theorem is actually present in this PDF
  (rather than under ``blanket statements''),
  that is actually an indication that it may be fairly obscure:
  the most important results are covered by the blanket statements.

  We strongly anti-recommend using the TOTAL as any sort of study guide,
  or trying to memorize theorems that you see in TOTAL.
  That would be a lot like trying to learn English as a second language
  by memorizing the Merriam-Webster dictionary.

  \item The TOTAL is meant to cover 99\% of the common situations, not 100\%.
  We don't think it's prudent to try to anticipate literally everything,
  and we leave the extremely rare corner cases up to grader discretion.
\end{itemize}

\subsection{Contributing}
If your favorite theorem is not included here,
you may submit a pull request on GitHub to consider it for inclusion.
See \href{https://github.com/omegausa/total/blob/main/CONTRIBUTING.md}{CONTRIBUTING.md}
for guidelines.

\subsection{Explicitly forbidden theorems}
\label{sec:forbidden}
In rare cases, OMEGA may decide to reject a request to add a certain theorem to the TOTAL.
A list of such cases, and their accompanying pull requests, are listed here:
\begin{quote}
  \emph{No forbidden results yet.}
\end{quote}

\newpage

\section{General blanket statements}
Any theorem appearing in any of the following textbooks may be quoted:
\begin{itemize}
  \item The OMEGA Syllabus
  \item The Art and Craft of Problem Solving (Paul Zeitz)
  \item The Art of Problem Solving Volumes 1 and 2 (Sandor Lehoczky and Richard Rusczyk)
  \item Chapter 2: Basic Concepts and Facts from The IMO Compendium
    (Thomas Mildorf)
\end{itemize}

\section{Additional theorems in algebra}
As a blanket statement, we allow any results in
\emph{Olympiad Inequalities} by Thomas Mildorf\footnote{This includes the $n-1$ equal
  value principle mentioned in the footnote on page 15.
  We include a caution that this comment is only for functions
  whose domain is $\mathbb R$, and not say $\mathbb R^+$.}

\subsection{Functional equations}
\begin{theorem}
  Suppose $f \colon \RR \to \RR$ satisfies $f(x+y) = f(x) + f(y)$.
  Then $f(qx) = qf(x)$ for any $q \in \QQ$.
  Moreover, $f$ is linear if there exists $(a,b) \in \RR^2$ and $\eps > 0$
  such that $(x-a)^2 + (f(x)-b)^2 > \eps$ for every $x$.
\end{theorem}

\subsection{Polynomials}
\begin{theorem}
  [Mason-Stothers theorem]
  Let $a(t)$, $b(t)$, $c(t)$ be pairwise coprime polynomials over a field,
  not all constant, satisfying $a + b + c = 0$.
  Then
  \[ \max(\deg a, \deg b, \deg c) \le n_0(abc) - 1, \]
  where $n_0(f)$ denotes the number of distinct roots of $f$.
\end{theorem}

\begin{theorem}
  [Lagrange interpolation formula]
  Let $k$ be a field of characteristic zero.
  Given $n+1$ distinct pairs $(x_0, y_0)$, \dots, $(x_n, y_n)$ in $k^2$
  with $x_i$ pairwise distinct,
  the unique polynomial of degree at most $n$ passing through all of them is
  \[ P(x) = \sum_{i=0}^{n} y_i \prod_{\substack{0 \le j \le n \\ j \neq i}}
    \frac{x - x_j}{x_i - x_j}. \]
\end{theorem}

\begin{theorem}
  [Newton's forward difference formula]
  Define forward differences by $\Delta^0 f(x) = f(x)$
  and $\Delta^{k} f(x) = \Delta^{k-1} f(x+1) - \Delta^{k-1} f(x)$.
  Then for any polynomial $f$ of degree at most $n$,
  \[ f(x) = \sum_{k=0}^{n} \binom{x}{k} \Delta^k f(0). \]
\end{theorem}

\section{Additional theorems in combinatorics}
\subsection{Graph theory}
\begin{theorem}
  [Max flow min cut]
  In a flow network, the maximum value of a flow
  from source $s$ to sink $t$ equals the minimum capacity
  of an $s$-$t$ cut.
\end{theorem}

\begin{theorem}
  [Four color theorem]
  Every planar graph is $4$-colorable.
\end{theorem}

\subsection{Game theory}
\begin{theorem}
  [Sprauge-Grundy]
  Every impartial game under the normal play convention
  is equivalent to a Nim heap of some size $n \ge 0$.
  The Grundy value of a position is the minimum excludant (mex)
  of the Grundy values of the positions reachable in one move.
\end{theorem}

\subsection{Extremal combinatorics}
\begin{theorem}
  [Dilworth theorem]
  In any finite partially ordered set,
  the maximum size of an antichain equals
  the minimum number of chains needed to partition the set.
\end{theorem}

\begin{theorem}
  [Erdos-Szekeres]
  Every sequence of more than $(r-1)(s-1)$ distinct real numbers
  contains an increasing subsequence of length $r$
  or a decreasing subsequence of length $s$.
\end{theorem}

\subsection{Enumerative identities}
\begin{theorem}
  [Hook-length formula]
  The number of standard Young tableaux of shape $\lambda$ with $n = |\lambda|$ cells is
  \[ \frac{n!}{\prod_{u \in \lambda} h(u)}, \]
  where $h(u)$ is the hook length of cell $u$.
\end{theorem}

\section{Additional theorems in geometry}
\subsection{Blanket statements}
We allow with proof any results from
\emph{Euclidean Geometry in Mathematical Olympiads} by Evan Chen.

\subsection{Projective results}
\begin{theorem}
  [Desargue involution theorem]
  Let $ABCD$ be four points on a conic, and let $\ell$ be a line.
  Then the three pairs of points
  $(AB \cap \ell,\; CD \cap \ell)$,
  $(AC \cap \ell,\; BD \cap \ell)$,
  $(AD \cap \ell,\; BC \cap \ell)$
  are in involution.
  If $\ell$ meets the conic at points $P$ and $Q$,
  then $(P, Q)$ is also a pair in this involution.
\end{theorem}

\section{Additional theorems in number theory}
\subsection{Famous theorems}
\begin{theorem}
  [Prime number theorem for arithmetic progressions]
  For coprime positive integers $a$ and $q$,
  the number of primes $p \le x$ with $p \equiv a \pmod{q}$ is
  \[ (1+o(1)) \frac{x}{\varphi(q) \ln x} \]
  as $x \to \infty$.
  In particular, every such arithmetic progression contains infinitely many primes.
\end{theorem}

\begin{theorem}
  [Zsigmondy's theorem]
  For coprime integers $a > b > 0$ and $n \ge 3$,
  the number $a^n - b^n$ has a prime divisor
  that does not divide $a^k - b^k$ for any $1 \le k < n$,
  with the sole exception $(a, b, n) = (2, 1, 6)$.
\end{theorem}

\begin{theorem}
  [Kobayashi's theorem]
  Let $S$ be an infinite set of integers.
  Then there is at most one integer $a \in \ZZ$ such that
  \[ \left\{ p \text{ prime} : p \mid s + a \text{ for some} s \in S\right\} \]
  is finite.
\end{theorem}

\begin{theorem}
  [Bertrand's postulate]
  For every integer $n \ge 2$, there exists a prime $p$ with $n < p < 2n$.
\end{theorem}

\begin{theorem}
  [Bounded gaps]
  There are infinitely many pairs of primes differing by at most $246$.
\end{theorem}

\begin{theorem}
  [Falting's theorem]
  An algebraic curve of genus $g \ge 2$ over a number field
  has only finitely many rational points.
\end{theorem}

\begin{theorem}
  [Catalan's conjecture]
  The only solution in integers $x, y, a, b > 1$
  to $x^a - y^b = 1$ is $3^2 - 2^3 = 1$.
\end{theorem}

\begin{theorem}
  [Vinogradov's theorem]
  Every sufficiently large odd integer
  can be expressed as the sum of three primes.
\end{theorem}

\begin{theorem}
  [Chen's theorem]
  Every sufficiently large even integer
  can be written as the sum of a prime
  and a semiprime (a product of at most two primes).
\end{theorem}

\begin{theorem}
  [Fermat's last theorem]
  There are no solutions in positive integers
  to $a^n + b^n = c^n$ for any integer $n \ge 3$.
\end{theorem}

\subsection{Modular arithmetic}
\begin{theorem}
  [Thue's lemma]
  For a prime $p$ and integer $a$ with $p \nmid a$,
  there exist integers $x, y$ with $1 \le x, y \le \lfloor \sqrt{p} \rfloor$
  such that $ax \equiv \pm y \pmod{p}$.
\end{theorem}

\subsection{Integer polynomials}
\begin{theorem}
  [Cohn criteria]
  Let $f(x) = a_n x^n + \dots + a_1 x + a_0$ with $a_i \in \ZZ$
  and $0 \le a_i \le b - 1$ for some integer $b \ge 2$.
  If $f(b)$ is prime, then $f(x)$ is irreducible over $\ZZ$.
\end{theorem}

\begin{theorem}
  [Hensel lemma]
  Let $f(x) \in \ZZ[x]$ and let $p$ be a prime.
  If $a \in \ZZ$ satisfies $f(a) \equiv 0 \pmod{p^k}$ and $p \nmid f'(a)$,
  then there is a unique $b \pmod{p^{k+1}}$ with $b \equiv a \pmod{p^k}$
  and $f(b) \equiv 0 \pmod{p^{k+1}}$.
\end{theorem}

\subsection{Results relating to $\nu_p$}
\begin{theorem}
  [Legendre formula]
  For a prime $p$ and positive integer $n$,
  \[ \nu_p(n!) = \sum_{i=1}^{\infty} \left\lfloor \frac{n}{p^i} \right\rfloor
    = \frac{n - s_p(n)}{p-1}, \]
  where $s_p(n)$ is the sum of the digits of $n$ in base $p$.
\end{theorem}

\begin{theorem}
  [Lifting the exponent lemma]
  Let $p$ be an odd prime and let $a, b$ be integers with
  $p \mid a - b$ but $p \nmid a$ and $p \nmid b$.
  Then\footnote{This actually appears in the IMO Compendium already,
  but the statement is so different-looking we provide it here again.}
  \[ \nu_p(a^n - b^n) = \nu_p(a - b) + \nu_p(n). \]
  Similarly, if $p \mid a + b$ and $n$ is odd,
  then $\nu_p(a^n + b^n) = \nu_p(a + b) + \nu_p(n)$.

  When $p=2$, we instead have the following result for all odd integers $x$ and $y$:
  \[
    \nu_2(x^n-y^n)
    = \begin{cases}
      \nu_2(x-y) + \nu_2(x+y) + \nu_2(n) - 1 & n \text{ even} \\
      \nu_2(x-y) & n \text{ odd}.
    \end{cases}
  \]
\end{theorem}

\subsection{Numerical semigroups}
\begin{theorem}
  [Chicken McNugget theorem]
  For coprime positive integers $m$ and $n$,
  the greatest integer that cannot be represented as $am + bn$
  for nonnegative integers $a, b$ is $mn - m - n$.
\end{theorem}


\section{Additional theorems in linear algebra}
\begin{theorem}
  [Spectral theorem for Hermitian matrices on $\mathbb{C}$]
  Every Hermitian matrix can be unitarily diagonalized.
  Equivalently, it admits an orthonormal basis of eigenvectors,
  and all eigenvalues are real.
\end{theorem}

\begin{theorem}
  [Cayley-Hamilton theorem]
  Every square matrix over a commutative ring
  satisfies its own characteristic polynomial.
\end{theorem}

\begin{theorem}
  [Farkas lemma]
  Let $A$ be an $m \times n$ real matrix and $b \in \RR^m$.
  Then exactly one of the following holds:
  \begin{enumerate}
    \item There exists $x \in \RR^n$ with $Ax = b$ and $x \ge 0$.
    \item There exists $y \in \RR^m$ with $A^\top y \ge 0$ and $b^\top y < 0$.
  \end{enumerate}
\end{theorem}

\section{Additional theorems in abstract algebra}
\begin{theorem}
  [Burnside lemma]
  The number of orbits of a finite group $G$ acting on a set $X$ is
  \[ \frac{1}{|G|} \sum_{g \in G} |X^g|, \]
  where $X^g = \{x \in X : g \cdot x = x\}$ is the set of elements fixed by $g$.
\end{theorem}

\begin{theorem}
  [Sylow theorem]
  Let $G$ be a finite group of order $p^a m$ where $p \nmid m$.
  \begin{enumerate}
    \item There exists a subgroup of $G$ of order $p^a$ (a Sylow $p$-subgroup).
    \item All Sylow $p$-subgroups are conjugate to each other.
    \item The number $n_p$ of Sylow $p$-subgroups satisfies
      $n_p \equiv 1 \pmod{p}$ and $n_p \mid m$.
  \end{enumerate}
\end{theorem}

\begin{theorem}
  [Fundamental theorem of Galois theory]
  Let $L/K$ be a finite Galois extension with Galois group $G$.
  There is an inclusion-reversing bijection between
  intermediate fields $K \subseteq E \subseteq L$
  and subgroups $H \le G$,
  given by $E \mapsto \operatorname{Gal}(L/E)$ and $H \mapsto L^H$.
  Moreover, $E/K$ is a normal extension
  if and only if the corresponding subgroup is normal in $G$.
\end{theorem}

\begin{theorem}
  [Chebotarev density theorem]
  Let $L/K$ be a Galois extension of number fields with Galois group $G$.
  For any conjugacy class $C \subseteq G$,
  the set of unramified primes of $K$ whose Frobenius conjugacy class equals $C$
  has natural density $|C|/|G|$.
\end{theorem}

\section{Additional theorems in real and complex analysis}
\begin{theorem}
  [Picard]
  A nonconstant entire function $f \colon \CC \to \CC$
  takes every complex value with at most one exception.
\end{theorem}

\begin{theorem}
  [Taylor series for $\sin$ and $\cos$]
  The following two Taylor series provide analytic extensions of the sine and cosine
  functions from $\CC$ to $\CC$:
  \begin{align*}
    \cos z &= 1 - \frac{z^2}{2!} + \frac{z^4}{4!} - \frac{z^6}{6!} + \dots \\
    \sin z &= z - \frac{z^3}{2!} + \frac{z^5}{5!} - \frac{z^7}{7!} + \dots.
  \end{align*}
\end{theorem}

\begin{theorem}
  [Extreme value theorem on compact sets]
  A continuous real-valued function on a nonempty compact set
  attains its maximum and minimum values.
\end{theorem}

\begin{theorem}
  [Heine--Borel]
  A subset of $\RR^n$ is compact
  if and only if it is closed and bounded.
\end{theorem}

\begin{theorem}
  [Bolzano--Weierstrass]
  Every bounded sequence in $\RR^n$
  has a convergent subsequence.
\end{theorem}

\begin{theorem}
  [Weierstrass approximation theorem]
  Let $f \colon [a,b] \to \RR$ be continuous.
  For every $\eps > 0$, there exists a polynomial $p$ such that $|f(x)-p(x)| < \eps$
  for all $x \in [a,b]$.
\end{theorem}

\begin{theorem}
  [Arzela--Ascoli theorem]
  A sequence of real-valued functions on a compact metric space
  has a uniformly convergent subsequence
  if and only if it is uniformly bounded and equicontinuous.
\end{theorem}

\begin{theorem}
  [Rouche's theorem]
  If $f$ and $g$ are holomorphic inside and on a simple closed contour $C$,
  and $|g(z)| < |f(z)|$ on $C$,
  then $f$ and $f + g$ have the same number of zeros inside $C$,
  counted with multiplicity.
\end{theorem}


\end{document}
